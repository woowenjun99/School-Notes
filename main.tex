% Edited: https://github.com/jovyntls/cheatsheets/blob/master/CS3230/cs3230-cheatsheet.tex
\documentclass[10pt, landscape]{article}
\usepackage[scaled=0.92]{helvet}
\usepackage{calc}
\usepackage{multicol}
\usepackage[a4paper,margin=3mm,landscape]{geometry}
\usepackage{amsmath,amsthm,amsfonts,amssymb}
\usepackage{color,graphicx,overpic}
\usepackage{hyperref}
\usepackage{newtxtext} 
\usepackage{enumitem}
\usepackage[table]{xcolor}
\usepackage{mathtools}
\setlist{nosep}

% tightcenter
\newenvironment{tightcenter}{%
  \setlength\topsep{0pt}
  \setlength\parskip{0pt}
  \begin{center}
    }{%
  \end{center}
}

% boxed
\newenvironment{tightbox}{%
  \setlength\topsep{0pt}
  \setlength\parskip{0pt}
  \begin{center}
    \begin{tabular}{|@{\hspace{\dimexpr\fboxsep+0.5\arrayrulewidth}}c@{\hspace{\dimexpr\fboxsep+0.5\arrayrulewidth}}|}
      \hline
    }
    {%
    \\ \hline
    \end{tabular}
  \end{center}
}

% fixed width box
\newenvironment{fixedbox}[1][0.7]{
  \setlength\topsep{0pt}
  \setlength\parskip{0pt}
  \begin{center}
    \begin{tabular}{|>{\centering\arraybackslash}m{#1\linewidth}|}
    \hline
  }{
  \\ \hline
  \end{tabular}
  \end{center}
}

% definition of a new term
\usepackage{soul}
\definecolor{paleyellow}{RGB}{251,243,218}
\newcommand{\definition}[2][]{\sethlcolor{paleyellow}\hl{\textbf{#2}} #1  $\rightarrow$}
% inline definition
\newcommand{\ildefinition}[1]{\sethlcolor{paleyellow}\hl{\textbf{#1}}}

% important note (attention)
\newcommand{\attention}{{\color{red}\textbf{! }}}

% nice proof
\newenvironment{niceproof}[1][Proof]
{%
  \sbox0{\textit{#1}. }%
  \list{}{\labelwidth\wd0 \leftmargin\wd0 \labelsep 0pt }
\item[\usebox0]}
  {\endlist}

\usepackage{color, soul}
\usepackage{listings}
\usepackage{inconsolata}

% C++ Code
\lstdefinestyle{customcpp}{
  belowcaptionskip=1\baselineskip,
  breaklines=true,
  frame=L,
  xleftmargin=\parindent,
  language=C++,
  showstringspaces=false,
  basicstyle=\footnotesize\ttfamily,
  keywordstyle=\bfseries\color{green!40!black},
  commentstyle=\itshape\color{purple!40!black},
  identifierstyle=\color{blue},
  stringstyle=\color{orange},
}

\definecolor{backcolour}{HTML}{F2F2F2}
\newcommand{\code}[1]{\texttt{\sethlcolor{backcolour}\hl{$\,$#1$\,$}}}
\pagestyle{empty}

% For code

\makeatletter
\renewcommand{\section}{\@startsection{section}{1}{0mm}%
  {-1ex plus -.5ex minus -.2ex}%
  {0.5ex plus .2ex}%x
{\normalfont\large\bfseries}}
\renewcommand{\subsection}{\@startsection{subsection}{2}{0mm}%
  {-1explus -.5ex minus -.2ex}%
  {0.5ex plus .2ex}%
{\normalfont\normalsize\bfseries}}
\renewcommand{\subsubsection}{\@startsection{subsubsection}{3}{0mm}%
  {-1ex plus -.5ex minus -.2ex}%
  {1ex plus .2ex}%
{\normalfont\small\bfseries}}%
\makeatother

\renewcommand{\familydefault}{\sfdefault}
\renewcommand\rmdefault{\sfdefault}
%  makes nested numbering (e.g. 1.1.1, 1.1.2, etc)
\renewcommand{\labelenumii}{\theenumii}
\renewcommand{\theenumii}{\theenumi.\arabic{enumii}.}
\renewcommand\labelitemii{•}
\renewcommand\labelitemiii{•}

\definecolor{mathblue}{cmyk}{1,.72,0,.38}
\everymath\expandafter{\the\everymath \color{mathblue}}

% Don't print section numbers
% \setcounter{secnumdepth}{0}

\setlength{\parindent}{0pt}
\setlength{\parskip}{0pt plus 0.5ex}
%% adjust spacing for all itemize/enumerate
\setlength{\leftmargini}{0.5cm}
\setlength{\leftmarginii}{0.5cm}
\setlist[itemize,1]{leftmargin=2mm,labelindent=1mm,labelsep=1mm}
\setlist[itemize,2]{leftmargin=3mm,labelindent=1mm,labelsep=1mm}
\setlist[itemize,3]{leftmargin=3mm,labelindent=1mm,labelsep=1mm}


\begin{document}
\raggedright
\footnotesize
\begin{multicols*}{4}
\setlength{\columnseprule}{0.25pt}
 \begin{center}
    \fbox{%
      \parbox{0.8\linewidth}{\centering \textcolor{black}{
          {\Large\textbf{CS3211}}
        \\ \normalsize{Parallel And Concurrent Programming}}
        \\ {\footnotesize \textcolor{gray}{github/woowenjun99}}
      }%
    }
  \end{center}
\section{Concurrency}
\begin{itemize}
    \item Two or more tasks can start, run or complete in overlapping time periods.
    \item \definition{Process} A program in the operating systems.
    \item \definition{Threads} Independent control flows of the program that share the same memory address in the process.
    \item \definition{Starvation} A situation where a process is prevented from making progress because some other resources has the resource it requires.
\end{itemize}

\subsection{Race Condition}
\begin{itemize}
    \item \definition{Race Condition} Two concurrent threads or processes access a shared resource without any synchronisation, and at least one thread modifies the shared resource.
    \item \definition{Data Race} Two accesses to a single memory location, no enforced ordering between the access, one or both accesses is a write and causes undefined behavior.
\end{itemize}

\subsection{Deadlock Conditions}
\begin{itemize}
    \item \definition{Coffman Condition} 4 conditions for a deadlock
    \begin{enumerate}
        \item Mutual Exclusion: At least one of the resource must be held in a non-shareable mode.
        \item Resource Holding: One process is holding one resource and requesting for another resource.
        \item No preemption: Resources cannot be preempted and critical sections cannot be aborted internally.
        \item Circular Wait: There must exist a set of processes $\{p_1, p_2, ..., p_n\}$ such that $p_1$ is waiting for $p_2$, $p_2$ is waiting for $p_3$ etc
    \end{enumerate}
\end{itemize}

\subsection{Parallelism}
\begin{itemize}
    \item \definition{Task Parallelism} Same types of tasks are assigned to the same thread. This leads to increase in separation of concerns as each thread is responsible for a stage of a pipeline. Example: Go pipeline concurrency pattern

    \item \definition{Data Parallelism} Each thread manages its own set of data. Example: Hand-over-hand lock
\end{itemize}

\section{Threads}
\begin{itemize}
    \item If we need to wait for a thread to finish, we use \code{join}. Otherwise, we use \code{detach}.
    \item If we intend to \code{join} the thread, we need to make sure to do so even when there is an exception. 
    If we intend to \code{detach} the thread, we need to be very careful when passing in arguments as their lifetime might have ended but the thread is still running.
\end{itemize}

\subsection{RAII}
\begin{itemize}
    \item \definition{Owner} An object containing a pointer to an object allocaated by new, for which delete is required. Every object in the free store must have exactly one owner.
    \item \definition{Resource Acquisition Is Initialisation} Binds the resource that must be acquired before use to the lifetime of an object.
    \item Threads instances are movable but not copyable.
\end{itemize}

\section{Concurrency In Go}
\begin{itemize}
    \item The Go programming language is created based on the principle of CSP.
    \item \definition{Communicating Sequntial Processes (CSP)} The output for one process is the input of another process.
    \item \definition{Goroutines} Functions that runs independently and follow the \code{fork-join} model similar to threads, but are not threads.
\end{itemize}

\section{Patterns In Go}
\subsection{Confinement}
\begin{itemize}
    \item Synchronise via communicating (e.g. channels)
    \item Data is protected by confinement
\end{itemize}
\subsubsection{Lexical Confinement (1)}

\begin{itemize}
    \item We can expose the reading or writing handle of the channel.
\end{itemize}

\begin{lstlisting}[language=Go, style=customcpp]
producer := func() <-chan int {
    results := make(chan int, 5)
    go func() {
        defer close(results)
        for i := 0; i < 5; ++i {
            results <- i
        }
    }()
    return results
}

consumer := func(results chan int) {
    for result := range(results) {
        fmt.Println(result)
    }
    fmt.Println("Done receiving")
}

consumer(producer())
\end{lstlisting}

\subsubsection{Lexical Confinement (2)}
\begin{itemize}
    \item Alternatively, we can expose only a slice of the array.
\end{itemize}
\begin{lstlisting}[language=Go, style=customcpp]
data := byte("golang")
go printData(data[:3])
go printData(data[3:])
\end{lstlisting}

\subsection{For-Select Pattern}
\begin{lstlisting}[language=Go, style=customcpp]
for {
    select{
    }
}
\end{lstlisting}
\begin{itemize}
    \item Sending iteration variable out of a channel
    \item Looping and waiting to be stopped. Since goroutines use resources, we need to ensure that it gets terminated. 
    \item The convention is that if a goroutine is responsible for creating a goroutine, it is responsible for ensuring it can stop the goroutine.
\end{itemize}

\end{multicols*}
\end{document}
